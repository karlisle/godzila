%%%%%%%%%%%%%%%%%%%%%%%%%%%%%%%%%%%%%%%%%
% Simple Sectioned Essay Template
% LaTeX Template
%
% This template has been downloaded from:
% http://www.latextemplates.com
%
% Note:
% The \lipsum[#] commands throughout this template generate dummy text
% to fill the template out. These commands should all be removed when 
% writing essay content.
%%%%%%%%%%%%%%%%%%%%%%%%%%%%%%%%%%%%%%%%%

%----------------------------------------------------------------------------------------
%	PACKAGES AND OTHER DOCUMENT CONFIGURATIONS
%----------------------------------------------------------------------------------------

\documentclass[12pt]{book} % Default font size is 12pt, it can be changed here
		\textheight = 28cm
		\textwidth = 19cm
		\topmargin = -1cm
		\oddsidemargin = -1cm
		\parindent = 5mm

\usepackage{geometry} % Required to change the page size to A4
\geometry{a4paper} % Set the page size to be A4 as opposed to the default US Letter

\usepackage{graphicx} % Required for including pictures

\usepackage{float} % Allows putting an [H] in \begin{figure} to specify the exact location of the figure
\usepackage{wrapfig} % Allows in-line images such as the example fish picture

%\usepackage{lipsum} % Used for inserting dummy 'Lorem ipsum' text into the template

\linespread{1} % Line spacing

%\setlength\parindent{0pt} % Uncomment to remove all indentation from paragraphs
\usepackage[utf8]{inputenc}
\usepackage[spanish]{babel}
\usepackage[T1]{fontenc}
\usepackage{fancyhdr}

% Configurar los encabezados, pies de pagina y paginas de capitulo
% Encabezados
\lhead[\thepage]{CAPÍTULO \thechapter. \rightmark}
\chead[]{}
\rhead[CAPÍTULO \thechapter. \leftmark]{\thepage}

\renewcommand{\headrulewidth}{0.5pt}

% Pie de pagina

\lfoot[]{\today}
\cfoot[]{}
\rfoot[J.Carlos Ávila]{}
\renewcommand{\footrulewidth}{0pt}

\usepackage{savesym}
\usepackage{amsmath}
\savesymbol{iint}
\usepackage{txfonts}
\restoresymbol{TXF}{iint}


\usepackage[x11names,table]{xcolor}
\usepackage{pstricks}
\usepackage[colorinlistoftodos, textwidth=2cm, shadow]{todonotes}
%\usepackage{hyperref}



\usepackage[colorlinks]{hyperref}
\usepackage[nogroupskip,nopostdot]{glossaries}
\setglossarystyle{altlist}
\makenoidxglossaries




%\usepackage[toc,style=altlistgroup,hyperfirst=false]{glossaries}


\hypersetup{
    colorlinks=true,
    linkcolor=rosa1,
    filecolor=magenta,      
    urlcolor=cyan,
}

\urlstyle{same}

\graphicspath{{./images/}} % Specifies the directory where pictures are stored

\definecolor{miorange}{rgb}{0.11, 0.43, 0.21}
\definecolor{rosa1}{RGB}{236, 46, 80}

%%%%%%%%%%%%%%%%%%%%%%%%%%%%%%%%%%%%%%%%%%%%%%%%%%%%%%%%%%%%%%%%%%%%%%%%%%%%%%%%%%%%%%%%%%%%%%%%%%%%%%%%%%%
%----------------------------------------------------------------------------------------
%	begin {Glosario}
%----------------------------------------------------------------------------------------

\newglossaryentry{os}{name={SO},description={Es el sistema o conjunto de aplicaciones que permiten que una computadora lleven a cabo sus funciones.}}


%%%%%%%%%%%%%%%%%%%%%%%%%%%%%%%%%%%%%%%%%%%%%%%%  %%%%%%%%%%%%%%%%%%%%%%%%%%%%%%%%%%%%%%%%%%%%%%%%%%%%%%%%%
\newglossaryentry{}{name={},description={}}
%%%%%%%%%%%%%%%%%%%%%%%%%%%%%%%%%%%%%%%%%%%%%%%%  %%%%%%%%%%%%%%%%%%%%%%%%%%%%%%%%%%%%%%%%%%%%%%%%%%%%%%%%%

%----------------------------------------------------------------------------------------
%	End {Glosario}
%----------------------------------------------------------------------------------------


\pagestyle{fancy}

\begin{document}

\lhead[\thepage]{CAPÍTULO \thechapter. \rightmark}
\rhead[CAPÍTULO \thechapter. \leftmark]{\thepage}


%----------------------------------------------------------------------------------------
%	TITLE PAGE
%----------------------------------------------------------------------------------------

\begin{titlepage}

\newcommand{\HRule}{\rule{\linewidth}{0.5mm}} % Defines a new command for the horizontal lines, change thickness here

\center % Center everything on the page
 
%----------------------------------------------------------------------------------------
%	HEADING SECTIONS
%----------------------------------------------------------------------------------------

\textsc{\LARGE Instituto Tecnológico de San Juan del Río}\\[1.5cm] % Name of your university/college
\textsc{\Large Centro de Investigación en ciencia Aplicada y Tecnología Avanzada}\\[0.5cm] % Major heading such as course name
%\textsc{\large Minor Heading}\\[0.5cm] % Minor heading such as course title

%----------------------------------------------------------------------------------------
%	TITLE SECTION
%----------------------------------------------------------------------------------------

\HRule \\[0.4cm]
{ \huge \bfseries Amplificación interactiva de contenido por medio de la detección de la dirección de la mirada.}\\[0.4cm] % Title of your document
\HRule \\[1.5cm]
 
%----------------------------------------------------------------------------------------
%	AUTHOR SECTION
%----------------------------------------------------------------------------------------

\begin{minipage}{0.4\textwidth}
\begin{flushleft} \large
\emph{Autor:}\\
J. Carlos \textsc{\'Avila Resendiz} % Your name
\end{flushleft}
\end{minipage}
~
\begin{minipage}{0.4\textwidth}
\begin{flushright} \large
\emph{Supervisor:} \\
Dr. Joaquin  \textsc{Salas Rodriguez} % Supervisor's Name
\end{flushright}
\end{minipage}\\[4cm]

% If you don't want a supervisor, uncomment the two lines below and remove the section above
%\Large \emph{Author:}\\
%John \textsc{Smith}\\[3cm] % Your name


%----------------------------------------------------------------------------------------
%	DATE SECTION
%----------------------------------------------------------------------------------------

{\large \today}\\[1cm] % Date, change the \today to a set date if you want to be precise

%----------------------------------------------------------------------------------------
%	LOGO SECTION
%----------------------------------------------------------------------------------------

\includegraphics{./imagenes/itsjr_s.jpg}\\ % Include a department/university logo - this will require the graphicx package

 
%----------------------------------------------------------------------------------------

\vfill % Fill the rest of the page with whitespace
\newpage
$\ $
\thispagestyle{empty}
\end{titlepage}

%----------------------------------------------------------------------------------------
%	TABLE OF CONTENTS
%----------------------------------------------------------------------------------------j

\pagenumbering{roman} 
\tableofcontents % Include a table of contents

\newpage % Begins the essay on a new page instead of on the same page as the table of contents
\thispagestyle{empty} 
%\appendix
%----------------------------------------------------------------------------------------
%	INTRODUCTION
%----------------------------------------------------------------------------------------
\pagenumbering{arabic}	
%\setcounter{page}{1}
	
			 
\newpage		 
\pagestyle{fancy}


\chapter{GENERALIDADES}
\thispagestyle{empty}
\markboth{GENERALIDADES}{GENERALIDADES}

\begin{minipage}{0.5\textwidth}
	\begin{flushleft} \large
	%\emph{•} \\
	\scriptsize	\textsl{\large “El auténtico genio consiste en la capacidad para evaluar información incierta, 
								aleatoria y contradictoria.”}\\
	\scriptsize \textbf{Winston Churchill, estadista.}
	\end{flushleft}
\end{minipage}\\[4cm]			

\newpage
\section{Objetivos}
	\subsection{Objetivo general}
	
		Desarrollar una aplicación de amplificación interactiva para computadoras con sistema operativo Windows, 
		que asista a personas con bajas capacidades visuales, por medio del seguimiento y estimación de la dirección 
		de la mirada sobre la pantalla de la computadora y en base a ello amplificar la zona de la pantalla en la 
		que enfoca la vista.
	
	\subsection{Objetivos específicos}
		Lograr la detección precisa y confiable del movimiento del globo ocular, con la ayuda de herramientas de
		procesamiento de imágenes.
		
		Hacer uso de las API's del sistema operativo que proveen las herramientas que  magnifican la zona de la 
		pantalla enfocada.
		

\newpage
\section{Justificación}

\newpage
\section{Caracterización de la empresa}
	\subsection{Datos generales de la empresa}
	
	\subsection{Descripción del departamento o área de trabajo}

\newpage
\section{Problemas a resolver}

\newpage
\section{Alcances y limitaciones}
	\subsection{Alcances}
	\subsection{Limitaciones}




\chapter{FUNDAMENTACIÓN TEÓRICA}
\markboth{FUNDAMENTACIÓN TEÓRICA}{FUNDAMENTACIÓN TEÓRICA} 
\thispagestyle{empty}

\section{Ingeniería del software}

\section{Herramientas de desarrollo}

\section{Lenguajes de programación}

\section{Metodologías de desarrollo de software}

\section{Eye Gaze }

\chapter{DESCRIPCIÓN DE ACTIVIDADES REALIZADAS}
\markboth{ACTIVIDADES REALIZADAS}{ACTIVIDADES REALIZADAS}
\thispagestyle{empty}
\newpage
\section{Análisis}

\section{Diseño}

\section{Desarrollo}

\section{Pruebas}

\section{Implementación}

\section{Retroalimentación}

\section{Resultados}

\section{Conclusiones y recomendaciones}

  	
%----------------------------------------------------------------------------------------
%	BEGIN GLOSARIO
%----------------------------------------------------------------------------------------			


%----------------------------------------------------------------------------------------
%	END GLOSARIO
%----------------------------------------------------------------------------------------

\section{Referencias Bibliográficas}
%---------------------------------------------------------------------------------------
%	BIBLIOGRAPHY
%----------------------------------------------------------------------------------------

\begin{thebibliography}{99} % Bibliography - this is intentionally simple in this template




%\bibitem{Luckham} 
%David Luckham
%[\textit{The Power of Events - An Introduction to Complex Event Processing in Distributed Enterprise Systems}]. 
%Addison-Wesley, ISBN 0-201-72789-7.


%\bibitem{Tello} 
%Adolfo Lozano Tello
%[\textit{Iniciación a la programación utilizando lenguajes visuales orientados a eventos.}] [ISBN978-84-7897-714-7]
%Ed.Bellisco Ediciones Técnicas y Científicas, ISBN 84-95279-49-5. ISBN 978-84-95279-49-1.


%\bibitem{} 
%Ricardo Barona Vázquez
%[\textit{Asterisk©, telefonía IP en software libre}].
%\href{Asterisk}{ http://www.enterate.unam.mx/artic/2008/abril/art3.html}

\end{thebibliography}


%----------------------------------------------------------------------------------------
 
\end{document}

