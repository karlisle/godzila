%%%%%%%%%%%%%%%%%%%%%%%%%%%%%%%%%%%%%%%%%
% Simple Sectioned Essay Template
% LaTeX Template
%
% This template has been downloaded from:
% http://www.latextemplates.com
%
% Note:
% The \lipsum[#] commands throughout this template generate dummy text
% to fill the template out. These commands should all be removed when 
% writing essay content.
%%%%%%%%%%%%%%%%%%%%%%%%%%%%%%%%%%%%%%%%%

%----------------------------------------------------------------------------------------
%	PACKAGES AND OTHER DOCUMENT CONFIGURATIONS
%----------------------------------------------------------------------------------------

\documentclass[12pt]{article} % Default font size is 12pt, it can be changed here
		\textheight = 26cm
		\textwidth = 18cm
		\topmargin = -1cm
		\oddsidemargin = -1cm
		\parindent = 5mm

\usepackage{geometry} % Required to change the page size to A4
\geometry{a4paper} % Set the page size to be A4 as opposed to the default US Letter

\usepackage{graphicx} % Required for including pictures

\usepackage{float} % Allows putting an [H] in \begin{figure} to specify the exact location of the figure
\usepackage{wrapfig} % Allows in-line images such as the example fish picture

\usepackage{lipsum} % Used for inserting dummy 'Lorem ipsum' text into the template

\linespread{1.2} % Line spacing

%\setlength\parindent{0pt} % Uncomment to remove all indentation from paragraphs
\usepackage[utf8]{inputenc}
\usepackage[spanish]{babel}
\usepackage[T1]{fontenc}
\usepackage{fancyhdr}

% encabezados
\lhead[\thepage]{CAPÍTULO \thechapter. \rightmark}
\chead[]{}
\rhead[CAPÍTULO \thechapter. \leftmark]{\thepage}
\renewcommand{\headrulewidth}{0.5pt}

% pie de pagina
\lfoot[]{\today}
\cfoot[]{}
\rfoot[CICATA]{}
\renewcommand{\footrulewidth}{0pt}


\usepackage{savesym}
\usepackage{amsmath}
\savesymbol{iint}
\usepackage{txfonts}
\restoresymbol{TXF}{iint}


\usepackage[x11names,table]{xcolor}
\usepackage{pstricks}
\usepackage[colorinlistoftodos, textwidth=2cm, shadow]{todonotes}
%\usepackage{hyperref}



\usepackage[colorlinks]{hyperref}
\usepackage[nogroupskip,nopostdot]{glossaries}
\setglossarystyle{altlist}
\makenoidxglossaries




%\usepackage[toc,style=altlistgroup,hyperfirst=false]{glossaries}


\hypersetup{
    colorlinks=true,
    linkcolor=rosa1,
    filecolor=magenta,      
    urlcolor=cyan,
}

\urlstyle{same}

\graphicspath{{./images/}} % Specifies the directory where pictures are stored

\definecolor{miorange}{rgb}{0.11, 0.43, 0.21}
\definecolor{rosa1}{RGB}{236, 46, 80}










%%%%%%%%%%%%%%%%%%%%%%%%%%%%%%%%%%%%%%%%%%%%%%%%  %%%%%%%%%%%%%%%%%%%%%%%%%%%%%%%%%%%%%%%%%%%%%%%%%%%%%%%%%

\newglossaryentry{}{name={},description={}}

%----------------------------------------------------------------------------------------
%	En {Glosario}
%----------------------------------------------------------------------------------------




\begin{document}

%----------------------------------------------------------------------------------------
%	TITLE PAGE
%----------------------------------------------------------------------------------------

\begin{titlepage}

\newcommand{\HRule}{\rule{\linewidth}{0.5mm}} % Defines a new command for the horizontal lines, change thickness here

\center % Center everything on the page
 
%----------------------------------------------------------------------------------------
%	HEADING SECTIONS
%----------------------------------------------------------------------------------------

\textsc{\LARGE Instituto Tecnológico de San Juan del Río}\\[1.5cm] % Name of your university/college
\textsc{\Large S.P.R.E.S.A.C}\\[0.5cm] % Major heading such as course name
%\textsc{\large Minor Heading}\\[0.5cm] % Minor heading such as course title

%----------------------------------------------------------------------------------------
%	TITLE SECTION
%----------------------------------------------------------------------------------------

\HRule \\[0.4cm]
{ \huge \bfseries Sistema para el registro de entradas y salidas de areas criticas}\\[0.4cm] % Title of your document
\HRule \\[1.5cm]
 
%----------------------------------------------------------------------------------------
%	AUTHOR SECTION
%----------------------------------------------------------------------------------------

\begin{minipage}{0.4\textwidth}
\begin{flushleft} \large
\emph{Autor:}\\
J. Carlos \textsc{\'Avila Resendiz} % Your name
\end{flushleft}
\end{minipage}
~
\begin{minipage}{0.4\textwidth}
\begin{flushright} \large
\emph{Supervisor:} \\
Dr. Joaquin  \textsc{Salas Rodriguez} % Supervisor's Name
\end{flushright}
\end{minipage}\\[4cm]

% If you don't want a supervisor, uncomment the two lines below and remove the section above
%\Large \emph{Author:}\\
%John \textsc{Smith}\\[3cm] % Your name

%----------------------------------------------------------------------------------------
%	DATE SECTION
%----------------------------------------------------------------------------------------

{\large \today}\\[1.5cm] % Date, change the \today to a set date if you want to be precise

%----------------------------------------------------------------------------------------
%	LOGO SECTION
%----------------------------------------------------------------------------------------


\includegraphics{./imagenes/itsjr_s.jpg}\\ % Include a department/university logo - this will require the graphicx package
 
%----------------------------------------------------------------------------------------

\vfill % Fill the rest of the page with whitespace

\end{titlepage}

%----------------------------------------------------------------------------------------
%	TABLE OF CONTENTS
%----------------------------------------------------------------------------------------j

\pagenumbering{roman} 
\tableofcontents % Include a table of contents

\newpage % Begins the essay on a new page instead of on the same page as the table of contents 
%\appendix
%----------------------------------------------------------------------------------------
%	INTRODUCTION
%----------------------------------------------------------------------------------------
\pagenumbering{arabic}	
\setcounter{page}{1}
	
	\begin{minipage}{0.5\textwidth}
		\begin{flushleft} \large
		%\emph{•} \\
		\scriptsize	\textsl{\large “El auténtico genio consiste en la capacidad para evaluar información incierta, aleatoria y contradictoria.”}\\
		\scriptsize \textbf{Winston Churchill, estadista.}
		\end{flushleft}
	\end{minipage}\\[4cm]
			
		 
\newpage		 
     

\section*{Introducción}

	El presente documento esta enfocado únicamente en los resultados del estudio del código existente del proyecto que da nombre 
    a este documento, \texttt{Amplificación interactiva de contenido por medio de detección de la dirección de la mirada}.
    
    Dado que en el trabajo previo no se incluyo documentación alguna de sobre la estructura del proyecto y/o de análisis de los requerimientos
    del sistema, es por ello que antes de seguir con el proyecto se hace necesario el documentar el trabajo existente.

\section{Visión artificial \label{vision_artificial}}
	La visión artificial o visión por computadora es la ciencia y la tecnología que permiten a las máquinas ver o extraer información de las 
    imágenes digitales y resolver alguna tarea o entender la escena que están visionando.
    este conjunto de técnicas nos permiten diseñar interacciones, de modo que el usuario utiliza su movimiento o manipulación de objetos para 
    interactuar con la aplicación.
    
    \subsection{Espectros de luz}
    	La 
	
\section{Seguimiento ocular \label{eyeGaze}}
	El seguimiento de ojos (traducido del inglés eye tracking) es el proceso de evaluar, bien el punto donde se fija la mirada (donde 
    estamos mirando), o el movimiento del ojo en relación con la cabeza. Este proceso es utilizado en la investigación en los sistemas
    visuales, en psicología, en lingüística cognitiva y en diseño de productos.
   	\footnote{\href{https://es.wikipedia.org/wiki/Seguimiento_de_ojos}{Seguimiento ocular: Wikipedia}}
    
    \subsection{Tipos de seguimiento}
    	\subsubsection{Mediante métodos invasivo}
        	Utilizando algo adjunto al ojo como una lente de contacto especial con un espejo incorporado o un sensor de campo magnético.
            El movimiento de la unión se mide con el supuesto que no se deslice de manera significativa cuando el ojo gire. Las mediciones
            realizadas con lentes de contacto han aportado grabaciones extremadamente detalladas de los movimientos oculares. Las bobinas
            magnéticas es el método que utilizan para realizar estudios sobre la dinámica y la fisiología subyacente al movimiento del ojo.
      	
        \subsubsection{Mediante métodos no invasivo}
        	El segundo tipo de seguimiento sería sin necesidad que haya contacto. A través de la luz, por lo general luz infrarroja, se
            refleja en los ojos y se capta mediante una cámara de vídeo o algún otro sensor óptico. La información recogida se analiza para
            extraer la rotación de los ojos y los cambios en los reflejos.
            
            Los métodos ópticos, especialmente los basados en la grabación de vídeo son ampliamente utilizados para el seguimiento de la
            mirada y están bien considerados porque no son invasivos y el coste es bajo.

		\subsubsection{Mediante potenciales eléctricos}
        	Finalmente, el tercer tipo utiliza el potencial eléctrico medido con electrodos colocados alrededor de los ojos para detectar el
            movimiento. Los ojos son el origen de un constante campo de potencial eléctrico que también se puede detectar en total oscuridad
            aunque estos estén cerrados. Puede estar modelado para generar un dipolo con el polo positivo en la córnea y el polo negativo en
            la retina.
            La señal eléctrica que se puede derivar al uso de dos pares de electrodos de contacto colocados en la piel alrededor del ojo se
            llama electrooculograma \footnote{\href{https://es.wikipedia.org/wiki/Electrooculograma}{Electrooculograma}} (EOG). 
            Si los ojos se mueven de la posición del centro hacia la periferia, la retina se acerca a uno de los electrodos, mientras que la
            córnea se acerca al opuesto.
            
            Este cambio en la orientación de los dipolos cambia consecuentemente los resultados del campo potencial eléctrico la señal
            \texttt{(EOG)} medida.
            
        \subsubsection{Tecnologías y técnicas}
        	La extensa mayoría de diseños actuales son seguidores de ojos basados en vídeos. Una cámara enfoca uno o los dos ojos y graba sus
            movimientos mientras el sujeto mira una serie de estímulos. Los seguidores de ojos más modernos usan el contraste para localizar
            el centro de la pupila y crear un reflejo de la córnea a través de luz infrarroja e infrarroja cercana no colimada.
            
            El vector entre estas dos características puede usarse para computar la intersección de la mirada con una superficie después de
            una simple calibración individual. Se usan dos tipos generales de técnicas de seguimiento de ojos: pupila brillante y pupila
            oscura. 
            La técnica de la pupila brillante genera un mejor contraste iris/pupila debido a un seguimiento de ojos más correcto en relación 
            a la pigmentación del iris y reduce significativamente les interferencias producidas por las pestañas y otras características
            ocultas.
            
             Esto además permite un seguimiento en condiciones que van desde la total oscuridad hasta una claridad alta. Estas técnicas no 
             son muy efectivas para hacer seguimientos en exteriores ya que se producen interferencias en su monitorización.
             
             Las configuraciones de los seguidores de ojos varían mucho; algunos se montan en la cabeza, otros requieren la cabeza solo para
             ser estables (por ejemplo con un apoya-mentón), y algunos siguen los movimientos de la cabeza de forma remota y automática
             durante el movimiento. La mayoría usan una frecuencia de muestreo de al menos 30 Hz. Aunque 50/60 Hz es lo más común, actualmente
             muchos seguimientos de ojos basados en vídeo funcionan a 240, 350 o incluso 1000/1250 Hz, frecuencia que se necesita para captar
             en detalle los rápidos movimientos durante la lectura o durante los estudios de neurología.
             
             El movimiento de los ojos normalmente se divide en fijaciones y salidas, cuando la mirada se detiene en cierta posición y cuando
             se mueve hacia otra posición respectivamente. Las series resultantes de las fijaciones y las salidas se llama \texttt{scanpath}.
             La mayoría de la información de los ojos se hace disponible durante la fijación, pero no durante la salida.
             
              Los uno o dos grados centrales \textit{(la fóvea)} aporta la mayor parte de información; los inputs de las excentricidades más
              extensas (la periferia) dan menos información. Por lo tanto, la localización de las fijaciones a lo largo del scanpath muestran
              que puntos de información de los estímulos son procesados durante una sesión de seguimiento de ojos. De media, las fijaciones 
              duran alrededor de 200ms durante la lectura de textos lingüísticos y 350ms durante la visión de una escena. Preparar la 
              salida hacia un nuevo objetivo lleva alrededor de 200ms.
              
\section{Análisis de requerimientos}
	\subsection{Funcionales}
    	El sistema debe ser capas de realizar todas y cada una de las siguientes acciones:
        \begin{itemize}
        	\item Detectar o reconocer los ojos.
            \item Determinar con alto nivel de exactitud la dirección de la mirada.
			\item Realizar el seguimiento de la dirección de la mirada.
            \item Efectuar la amplificación del área de la pantalla en la cual se esta enfocando la mirada.
            \item Tener una interfaz de configuración, que contemple lo siguiente:
            	\begin{enumerate}
                	\item Calibración inicial.
					\item Entrenamiento del usuario.
                    \item Configuraciones personalizadas.
				\end{enumerate}
            \item En su version inicial debe correr en computadoras con sistema operativo Windows 
            \footnote{\scriptsize Pese a que ese es el requerimiento inicial, se busca que sea multi-plataforma y en especial enfocado a
            dispositivos móviles}.
		\end{itemize}
       
       Cumplidos estos requisitos definidos con anterioridad se considera que el prototipo funcional para maquinas con \texttt{SO} Windows esta
       completo.
       
	\subsection{No funcionales}
    	
    	\begin{itemize}
        	\item Debe ejecutarse en maquinas con características básicas \footnote{\scriptsize Consideramos características básicas un
            procesador con al menos dos núcleos y 1Gb de memoria RAM}.
			\item Debe trabajar en imágenes de baja resolución.
            \item Debe ejecutarse en tiempo real.
            \item Debe ser lo suficientemente fluido \textit{optimizado} al realizar el seguimiento.
		\end{itemize}
        
	
\section{Diseño}
	Los componentes que se describirán a continuación son única y expresamente los que ya existen en el proyecto existente y sobre el cual se 
    se esta trabajando, y que la finalidad del mismo es tener un documento que describa de la manera mas certera posible el funcionamiento 
    tanto individual como en conjunto.
    
    \subsection{Módulos del sistema}
    	Por razones que escapan de mi conocimiento, no hay documentación de ningún tipo ya sea técnica, de usuario o de análisis de factibilidad
        del proyecto, las siguientes secciones son de cierta forma meras interpretaciones en base al código existente del proyecto existente, por
        ahora solo con el único fin de tener una perspectiva clara de lo que ya existe y poder ver que partes de ello se pueden reutilizar en un
        futuro, ya sea parte del código o módulos completos.
        
        \subsubsection{Main Window}
        	Componentes de la clase:
        	\begin{itemize}
        		\item detect
        		\item predict
        		\item train
        		\item readSettings
        		\item preview
        		\item drawPose
        	\end{itemize}
		Es con esta clase con la que que se inicia el ciclo de ejecución del sistema, se encarga principalmente de detectar en primera instancia desde la
		disponibilidad de la cámara web o dispositivo que servirá como la fuente encargada de proporcionar las imágenes que se procesaran.
		\begin{description}
			\item[detect:] 
				\begin{itemize}
					\item Inicializar variables
					\item Opcionalmente se puede guardar una copia del stream proporcionado por el dispositivo de captura.
					\item Se cargan dos componentes de una librería externa encargada de detectar rasgos faciales y la posición de la cabeza.
						\begin{enumerate}
							\item \textit{DetectModel-v1.5.bin }
							\item \textit{TrackingModel-v1.10.bin }
						\end{enumerate}
						Si no retorna el descriptor de alineación se notifica el evento.
					\item Cargar los modelos en cascada para la detección de rostros.
						\begin{enumerate}
							\item \textit{haarcascade\_frontalface\_alt2.xml}
						\end{enumerate}
						Si no se encuentra se notifica.
					\item Se crean tres ventanas:
						\begin{enumerate}
							\item \textbf{Gaze:} muestra toda la cara, con dos puntos que indican la mirada en el centro del iris.
							\item \textbf{Lefteye:} una pequeño ventana que enmarca el ojo izquierdo.
							\item \textbf{Righteye:}  una pequeño ventana que enmarca el ojo derecho.
							\item Las mismas solo desaparecen hasta que el usuario presiona la tecla \textit{'Esc'}.
						\end{enumerate}
				\end{itemize}
		\end{description}
		
		\subsubsection{Descripción detallada de los métodos.}
		        
        \subsubsection{New Hough}
        	Componentes de la clase: \vspace{1cm}
            	\begin{itemize}
                	\item NewHouh
                    \item circle\_hough
                    \item circle\_houghpeaks
                    \item circle\_points
                    \item getLUT
                    \item getEllipseLUT
                    \item ellipse\_hough
                    \item ellipse\_houghpeaks
                    \item intersection
                    
				\end{itemize}

        \subsubsection{eyGaze}
        	Componentes de la clase: 
        	
            \begin{itemize}
				\item ComputeIrisCenter
				\item computeHeadPosition
				\item proyectionPlane
				\item computeIrisCenterEllipse
				\item getEquializedeye
				\item getEyeEdges
				\item getIrisRange
				\item getHoughAccumulator
			\end{itemize}
           
            

        \subsubsection{Trainer}
        %Su función principal es de entrenar tanto al algoritmo de reconocimiento en cascada de \texttt{OpenCV} como al usuario, en etapa de
        %desarrollo se usa para el entrenamiento del algoritmo con la finalidad de que aprenda a discriminar todo aquello que parece ser \'ojos\' de
        %los reales \footnote{\scriptsize \textit{Considero que lo ideal es separar estas dos características, por un lado un modulo que se encarge 
        %de calibrar el sistema y otro que entrene al usuario en el uso de la aplicación.s }}.
        Componentes de la clase:
        \begin{itemize}
        	\item addTrainer
        	\item trainSVModels
        	\item saveSVModels
        	\item loadSVModels
        	\item trainRegModels
        	\item saveRegModels
        	\item loadRegModels
        	\item predictPosition
        \end{itemize}
        
        
        \subsubsection{Addaptative Canny}
        Componentes de la clase:
        \begin{itemize}
        	\item adaptativeCanny
        	\item smoothGradient
        	\item selectThresholds
        \end{itemize}
   
	
\subsection{Conclusión}
  
  	
%----------------------------------------------------------------------------------------
%	BEGIN GLOSARIO
%----------------------------------------------------------------------------------------			
\newpage


\printnoidxglossaries
\pagenumbering{roman}
%----------------------------------------------------------------------------------------
%	END GLOSARIO
%----------------------------------------------------------------------------------------

\newpage

%---------------------------------------------------------------------------------------
%	BIBLIOGRAPHY
%----------------------------------------------------------------------------------------

\begin{thebibliography}{99} % Bibliography - this is intentionally simple in this template




\bibitem{Luckham} 
David Luckham
[\textit{The Power of Events - An Introduction to Complex Event Processing in Distributed Enterprise Systems}]. 
Addison-Wesley, ISBN 0-201-72789-7.


\bibitem{Tello} 
Adolfo Lozano Tello
[\textit{Iniciación a la programación utilizando lenguajes visuales orientados a eventos.}] [ISBN978-84-7897-714-7]
Ed.Bellisco Ediciones Técnicas y Científicas, ISBN 84-95279-49-5. ISBN 978-84-95279-49-1.


%\bibitem{} 
%Ricardo Barona Vázquez
%[\textit{Asterisk©, telefonía IP en software libre}].
%\href{Asterisk}{ http://www.enterate.unam.mx/artic/2008/abril/art3.html}

\end{thebibliography}


%----------------------------------------------------------------------------------------
 
\end{document}

